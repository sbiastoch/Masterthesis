\section*{Abstract}
%%%%%%%%%%%%%%%%%%%%%%
% Was ist das Thema? Welche Methoden? % Warum relevant? Wie Evaluiert? Was für Ergebnisse/Schlüsse?
%%%%%%%%%%%%%%%%%%%%%%
Relational database systems are a powerful tool when working with vast amounts of complex data. Most industries rely on big data for everyday businesses not to mention the recent ascend of machine learning. Usually, data is stored in databases and transferred into applications to perform complex computations. By leaving the database-world we sacrifice everything that makes databases fast: Query planer, indices, optimized I/O along with others. Furthermore, we have to accept the overhead of retrieving the raw-data from the database-system.

Many complex and interesting computations on big data share a common property that makes most programmers switch instantly from SQL to any other programming language: Recursion. Writing recursive algorithms as recursive queries can be very challenging, since most algorithms are designed in a functional or imperative style rather than data-oriented way. User Defined Functions (UDFs) were introduced to ease the development of complex queries and can even handle recursion -- but recursive UDFs are incredibly slow.

With our approach we give the programmer a fully automated tool that translates a recursive SQL-UDF easy to write and maintain into a single semantically equivalent query. The translation utilizes automatic memoization and removes the overhead from recursive UDFs. A first evaluation shows its efficiency for memoizable as well as divide-and-conquere algorithms. 

This thesis provides an operational semantics to analyze a given UDF statically and extract features required to construct the translation from a set of templates. It is accompanied by a first implementation in Haskell for PostgreSQL.