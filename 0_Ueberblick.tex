\section*{Abstract}
%%%%%%%%%%%%%%%%%%%%%%
% Was ist das Thema? Welche Methoden? % Warum relevant? Wie Evaluiert? Was für Ergebnisse/Schlüsse?
%%%%%%%%%%%%%%%%%%%%%%
Relational database systems are a powerful way of working with vast amounts of complex data. Most industries nowadays rely on big data for everyday businesses, not to mention the recent ascend of machine learning. Usually, data is stored in databases and transferred into applications to do complex computations. By leaving the database-world, we sacrifice everything that makes databases fast: Query planer, indices, optimized I/O and more. Furthermore, we have to take the overhead of retrieving the raw-data from the database-system.

Many complex and interesting computations on big data share a common property that makes most programmers switch instantly from SQL to any other programming language: Recursion. Writing recursive algorithms as recursive queries may be very challenging since most algorithms are designed in a functional or imperative style rather than data-oriented. User Defined Functions (UDFs) were introduced to ease development of complex queries and can even handle recursion, but recursive UDFs are incredibly slow.

With our approach we give the programmer a tool that can be used to translate a recursive function, given by an SQL-UDF that is easy to write and maintain, into a semantically equivalent single recursive query. The translation outperforms the original UDF by many magnitudes by utilizing all the features of the database-system and using memoization. This thesis provides an operational semantics to analyze a given UDF statically and extract features required to construct the translation. I give an exemplary implementation in Haskell for PostgreSQL.