\chapter{Conclusion and future work} \label{conclusion}
Analyzing the callstack to limit rows needed to carry around
Remove calls from the callstack when all dependant calls are evauated

\section{\texttt{STRICT} functions}
Functions tagged as \texttt{STRICT} are functions that return \texttt{NULL} straight away if they any of their arguments is \texttt{NULL}. The actual function is not evaluated in this case. In this case no call with a \textit{NULL} argument will lead to subsequent callsites, so that all recursive scenarios have non-\texttt{NULL} arguments. Calls with a \texttt{NULL}-argument are therefore only required to consider when collecting basecases. During evaluation all arguments are not nullable and \texttt{x IS NOT DISTINCT FROM y} can be simplified to \textit{x = y}. This may yield to a large performance gain as the implicit \texttt{x = y OR (x IS NULL AND y IS NULL)} is removed wich degrades join-performance.

\section{Closures of SQL-fragments containing row-variables}
\section{Tabular return types}
\section{Evaluation of predicate-parts in own CTE}
\section{Parallel execution of in dependant callstack-subtrees}
\section{Iterative recursive calls to limit size of evaluation-table}
\section{Indices}
\section{Translation from and to other languages}

Single callsite => no calsite id required

%%%%%%%%%%%%%%%%%%%%%%%%%
% Was war das Ziel?
% Was habe ich getan?
% Was wurde erreicht?
% Was lernen wir daraus?
% "Woran könnte man weiter forschen?"
%%%%%%%%%%%%%%%%%%%%%%%%%
\section*{Acknowledgements}
Thanks to Chris, Denis, Grust, proofreader, ...