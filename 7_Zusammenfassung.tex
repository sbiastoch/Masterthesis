\chapter{Conclusion} \label{conclusion}

%%%%%%%%%%%%%%%%%%%%%%%%%
% Was war das Ziel?
% Was habe ich getan?
% Was wurde erreicht?
% Was lernen wir daraus?
% "Woran könnte man weiter forschen?"
%%%%%%%%%%%%%%%%%%%%%%%%%


Goal of this thesis was to automatically translate recursive SQL-UDFs into single, declarative queries. Recursive text-book style UDFs are easy to implement but trade performance for readability. One the other hand, manual optimizations are a time-consuming and challenging task that trades readability for performance. With the presented approach the programmer can keep writing recursive UDFs and translate them fully automatically into a optimized version.

The first step of the translation is the scenario analysis that slices the original UDF into its different evaluation scenarios. Each scenario-query is accompanied by a predicate that is used to detect the appropriate scenario for each call.

The analysis works already on a wide range for UDFs with a few constraints. The provided rules are able to create scenarios for nested W-S-F-W queries that use \texttt{CASE}-expressions as conditional. Callsites may be located in either the select-, from- or where-clause as well as in CTEs. Neither predicates nor callsites may contain row-references from the surrounding query, limiting the use of tables. An other notable constraint is that the evaluation of a callsite may not depend on the result of an other callsite.

Analysis and template filling are well separated steps, allowing to be extended easily in the future. Analysis rules are implemented very closely to the formal rules. The implementation of the templates is also modular, which makes it easy to replace parts of the templates or introduce new optimizations. The implementation for PostgreSQL has been tested with 30+ UDFs ranging from very simple functions up to complicated functions with more than 10 scenarios and nested CTEs.

The optimized translation performs better than the original for all tested UDFs, except for very tiny inputs. The translation performs automatic memoization, yielding to a speed-up by magnitudes for problems with overlapping subproblems. For divide-and-conquere algorithms the performance is improved by approximately factor 10. Yet, manual optimizations greatly outperform automatic translations as expected.

The provided optimizations has been shown to be useful, especially the template for single recursive functions. Without the optimization for single recursive functions, the estimates of the query planner lead to a plan that performs worse than the original UDF. Other optimizations speed-up evaluation by a fair constant factor.

There are two promising directions for future work: Widening the range of translateable functions or optimize evaluation by exploiting information from the callgraph. Allowing callsites to iterate through rows from a table would be a very useful improvement, especially as the motivation for this thesis is to make it more practical to perform computations near the data. Runtimes could be improved by harvesting and exploiting the information encoded in the callgraph.

\section*{Acknowledgements}
A number of people have supported me not only throughout this thesis, but my entire studies. Most notably my parents, who enabled me to studing in Tübingen and abroad and always encouraged me to slow down and learn "non scholae sed vitae".

I want to thank my supervisor, Christian Duta, for the very good support throughout my thesis. Thanks to Denis Hirn, who helped me unraveling the internals of PostgreSQL and provided a great library to generate ASTs from queries and pretty-printing them. Thanks to Prof. Grust to arouse my excitement about databases and functional programming in many great and memorable lectures. 
Many thanks to my girlfriend Ingrid Starnecker for the continuous support, especially during this thesis. Special thanks to my proofreaders Michael Lieberum and Johannes Jakubeit.