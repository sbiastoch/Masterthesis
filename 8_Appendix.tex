\cleardoublepage
\renewcommand{\thesection}{\Alph{section}}%
%\appendix 
%\addcontentsline{toc}{chapter}{Anhang} 

\chapter[Appendix]{Appendix}
\phantomsection

\section{Usage of \texttt{twr}}
\begin{figure}[h!]
    \centering
    \scriptsize
\begin{verbatim}
Usage: twr (-f FILE | -d DIR | -u FUNC | -a) [-c] [-r] [-p] [-n FNAME/SUFFIX]
           [--user USER] [--password PASSWORD] [--host HOST] [--port PORT]
           [--database DATABASE] [-o PATH] [--disable-optimizations|--no-opt]
           [--disable-tail-recursion|--no-tr]
           [--disable-single-recursion|--no-sr]
           [--disable-constant-parameters|--no-cp] [-v] [-q|--quiet]
  Optimize recursive User-Defined Functions for PostgreSQL.

Available options:
  -f FILE                  Translate a UDF from a file
  -d DIR                   Translate UDFs that are given as single files in a
                           folder
  -u FUNC                  Translate an UDF by specifying the signature of an
                           existing function, eg. fib(int)
  -a                       Translate entire database. EXPERIMENTAL! Use with
                           caution!
  -c                       Persist translation to database
  -r                       Replace existing function in database
  -p                       Print translation(s) to stdout
  -n FNAME/SUFFIX          Name or prefix (for directory/db translation) of the
                           generated function.
  --user USER              Postgres username, default is ENV[USER]
  --password PASSWORD      Password for PostgreSQL user
  --host HOST              Host of PostgreSQL
  --port PORT              Port of PostgreSQL
  --database DATABASE      Name of a PostgreSQL database, default is USER
  -o PATH                  Create a file with the translated function.
  --disable-optimizations,--no-opt
                           Disable all optimizations
  --disable-tail-recursion,--no-tr
                           Do not use optimizations for tail recursion
  --disable-single-recursion,--no-sr
                           Do not use optimizations for single recursion
  --disable-constant-parameters,--no-cp
                           Disable optimizations for constant function
                           parameters
  -v                       Level of verbosity: -v = input/output info, -vv =
                           translation summary, -vvv = detailed analysis output
  -q,--quiet               No output
  -h,--help                Show this help text
\end{verbatim}
    \caption{Usage of the Haskell implementation \texttt{twr}}
\end{figure}

\section{Exemplary analysis results utilizing different rules}
\subsection{\REXPR-rule}
\begin{figure}[h!]
    \centering
    \begin{minted}{postgresql}
CREATE FUNCTION case_plus_case(n numeric) RETURNS numeric AS $$
SELECT CASE
     WHEN n < 1 THEN 1
     ELSE case_plus_case(n - 1)
   END
   +
   CASE
     WHEN n < 3 THEN 2
     WHEN n < 4 THEN case_plus_case(n - 2)
     ELSE 1 - case_plus_case(n - 3)
   END
$$ LANGUAGE SQL;
\end{minted}
    \caption{UDF \texttt{case\_plus\_case}, used to showcase application of the \REXPR-rule.}
    \label{udf:case_plus_case}
\end{figure}

\begin{figure}[h!]
    \centering
    \begin{lstlisting}[basicstyle=\ttfamily\scriptsize]
### Analysis result: ###
Name:              pg_proc.case_plus_case
Return type:       TAtom "numeric" [$\checkmark$ hashable]
Tail-Recursive:    $\times$
Single-Recursion:  $\times$
Parameter:
  1) TAtom "numeric" [$\checkmark$ hashable]
Constant Parameter:
   - none -
Callsites:
  1) Param 1:
        SELECT $\udfArg{1}$ - 2 
  2) Param 1:
        SELECT $\udfArg{1}$ - 3 
  3) Param 1:
        SELECT $\udfArg{1}$ - 1 
  4) Param 1:
        SELECT $\udfArg{1}$ - 1 
  5) Param 1:
        SELECT $\udfArg{1}$ - 2 
  6) Param 1:
        SELECT $\udfArg{1}$ - 1 
  7) Param 1:
        SELECT $\udfArg{1}$ - 3 
Recursive Scenarios:
  1) Callsites: 1
     Predicate:
        SELECT $\udfArg{1}$ < 1 AND (NOT $\udfArg{1}$ < 3 AND $\udfArg{1}$ < 4)
     Query:
        SELECT 1 + case_plus_case($\udfArg{1}$ - 2)

  2) Callsites: 2
     Predicate:
        SELECT $\udfArg{1}$ < 1 AND (NOT $\udfArg{1}$ < 3 AND NOT $\udfArg{1}$ < 4)) 
     Query:
        SELECT 1 + (1 - case_plus_case($\udfArg{1}$ - 3))

  3) Callsites: 3
     Predicate:
        SELECT NOT $\udfArg{1}$ < 1 AND $\udfArg{1}$ < 3
     Query:
        SELECT case_plus_case($\udfArg{1}$ - 1) + 2

  4) Callsites: 4, 5
     Predicate:
        SELECT NOT $\udfArg{1}$ < 1 AND (NOT $\udfArg{1}$ < 3 AND $\udfArg{1}$ < 4) 
     Query:
        SELECT case_plus_case($\udfArg{1}$ - 1) + case_plus_case($\udfArg{1}$ - 2)

  5) Callsites: 6, 7
     Predicate:
        SELECT NOT $\udfArg{1}$ < 1 AND (NOT $\udfArg{1}$ < 3 AND NOT $\udfArg{1}$ < 4)
     Query:
        SELECT case_plus_case($\udfArg{1}$ - 1) + (1 - case_plus_case($\udfArg{1}$ - 3)

Basecase Scenarios:
  1) Predicate:
        SELECT $\udfArg{1}$ < 1 AND $\udfArg{1}$ < 3 
     Query:
        SELECT 1 + 2
\end{lstlisting}
    \caption{Analysis output of \autoref{udf:case_plus_case}, for demonstration of the \REXPR-rule (slightly beautified).}
    \label{scenarios:case_plus_case}
\end{figure}
\FloatBarrier

\subsection{\RSELECT-rule}
\begin{figure}[h!]
    \centering
    \begin{minted}{postgresql}
CREATE FUNCTION fib_cols(n NUMERIC) RETURNS NUMERIC AS $$
SELECT fib.n_1 + fib.n_0 FROM (
  SELECT CASE WHEN n <= 2 THEN 0
              ELSE fib_cols(n - 1) END,
         CASE WHEN n <= 2 THEN 1
              ELSE fib_cols(n - 2) END
) AS fib(n_1, n_0)
$$ LANGUAGE SQL;
\end{minted}
    \caption{UDF \texttt{fib\_cols}, used to showcase application of the \RSELECT-rule.}
    \label{udf:fib_cols}
\end{figure}

\begin{figure}[h!]
    \centering
    \begin{lstlisting}[basicstyle=\ttfamily\scriptsize]
### Analysis result: ###
Name:              pg_proc.fib_cols
Return type:       TAtom "numeric" [$\checkmark$ hashable]
Tail-Recursive:    $\times$
Single-Recursion:  $\times$
Parameter:
  1) TAtom "numeric" [$\checkmark$ hashable]
Constant Parameter:
   - none -
Callsites:
  1) Param 1:
        SELECT $\udfArg{1}$ - 2 
  2) Param 1:
        SELECT $\udfArg{1}$ - 1 
  3) Param 1:
        SELECT $\udfArg{1}$ - 1 
  4) Param 1:
        SELECT $\udfArg{1}$ - 2 
Recursive Scenarios:
  1) Callsites: 1
     Predicate:
        SELECT (SELECT $\udfArg{1}$ <= 2 AND NOT $\udfArg{1}$ <= 2) 
     Query:
        SELECT ("subquery0"."n_1" + "subquery0"."n_0") :: numeric
        FROM (
          SELECT 0,
                 fib_cols($\udfArg{1}$ - 2)
        ) AS "subquery0"("n_1", "n_0")

  2) Callsites: 2
     Predicate:
        SELECT (SELECT NOT $\udfArg{1}$ <= 2 AND $\udfArg{1}$ <= 2) 
     Query:
        SELECT ("subquery0"."n_1" + "subquery0"."n_0") :: numeric
        FROM (
          SELECT fib_cols($\udfArg{1}$ - 1), 
                 1
        ) AS "subquery0"("n_1", "n_0")

  3) Callsites: 3, 4
     Predicate:
        SELECT (SELECT (NOT $\udfArg{1}$ <= 2 AND NOT $\udfArg{1}$ <= 2)) 
     Query:
        SELECT ("subquery0"."n_1" + "subquery0"."n_0") :: numeric
        FROM (
          SELECT fib_cols($\udfArg{1}$ - 1), 
                 fib_cols($\udfArg{1}$ - 2)
        ) AS "subquery0"("n_1", "n_0")

Basecase Scenarios:
  1) Predicate:
        SELECT (SELECT ($\udfArg{1}$ <= 2 AND $\udfArg{1}$ <= 2)) 
     Query:
        SELECT ("subquery0"."n_1" + "subquery0"."n_0") :: numeric
        FROM (
          SELECT 0,
                 1
        ) AS "subquery0"("n_1", "n_0")

\end{lstlisting}
    \caption{Analysis output of \autoref{udf:fib_cols}, for demonstration of the \RSELECT-rule (slightly beautified).}
    \label{scenarios:fib_cols}
\end{figure}
\FloatBarrier

\subsection{\RFROM-rule}
\begin{figure}[h!]
    \centering
    \begin{minted}{postgresql}
CREATE FUNCTION fib_from(n NUMERIC)
RETURNS NUMERIC AS $$
SELECT
  CASE
    WHEN n <= 2 THEN 1
    ELSE (
      SELECT T.i + S.i
      FROM
        (SELECT fib_from(n - 1)) AS T(i),
        (SELECT fib_from(n - 2)) AS S(i)
     )
END
$$ LANGUAGE SQL;
\end{minted}
    \caption{UDF \texttt{fib\_from}, used to showcase application of the \RFROM-rule.}
    \label{udf:fib_from}
\end{figure}

\begin{figure}[h!]
    \centering
    \begin{lstlisting}[basicstyle=\ttfamily\scriptsize]
### Analysis result: ###
Name:              pg_proc.fib_from
Return type:       TAtom "numeric" [$\checkmark$ hashable]
Tail-Recursive:    $\times$
Single-Recursion:  $\times$
Parameter:
  1) TAtom "numeric" [$\checkmark$ hashable]
Constant Parameter:
   - none -
Callsites:
  1) Param 1:
        SELECT (SELECT $\udfArg{1}$ - 1) 
  2) Param 1:
        SELECT (SELECT $\udfArg{1}$ - 2) 
Recursive Scenarios:
  1) Callsites: 1, 2
     Predicate:
        SELECT NOT $\udfArg{1}$ <= 2 AND (SELECT (SELECT (SELECT True)))
     Query:
        SELECT (SELECT "subquery0"."i" + "subquery1"."i" 
                 FROM (SELECT fib_from($\udfArg{1}$ - 1)) AS "subquery0"("i"), 
                      (SELECT fib_from($\udfArg{1}$ - 2)) AS "subquery1"("i")
               ) :: numeric

Basecase Scenarios:
  1) Predicate:
        SELECT $\udfArg{1}$ <= 2
     Query:
        SELECT 1
\end{lstlisting}
    \caption{Analysis output of \autoref{udf:fib_from}, for demonstration of the \RFROM-rule (slightly beautified).}
    \label{scenarios:fib_from}
\end{figure}

\FloatBarrier
\subsection{\RWHERE-rule}
\begin{figure}[h!]
    \centering
    \begin{minted}{postgresql}
CREATE FUNCTION sieve(i int, xs int []) RETURNS int [] AS $$
SELECT array_agg(T.v)
FROM unnest(xs) AS T(v)
WHERE CASE WHEN 2 * i > array_length(xs, 1) THEN TRUE
           ELSE (T.v = i -- keep current i
                 OR T.v % i != 0 -- remove all multiples of i
                ) AND T.v = ANY(sieve(i + 1, xs)) -- keep only other nonprime numbers
          END;
$$ LANGUAGE SQL;
\end{minted}
    \caption{UDF \texttt{sieve} implementing the Sieve of Eratosthenes, used to showcase application of the \RWHERE-rule.}
    \label{udf:sieve}
\end{figure}

\begin{figure}[h!]
    \centering
    \begin{lstlisting}[basicstyle=\ttfamily\scriptsize]
### Analysis result: ###
Name:              pg_proc.sieve
Return type:       TArray (TAtom "int4") [$\checkmark$ hashable]
Tail-Recursive:    $\times$
Single-Recursion:  $\checkmark$
Parameter:
  1) TAtom "int4" [$\checkmark$ hashable]
Constant Parameter:
  2) TArray (TAtom "int4") [$\checkmark$ hashable]
Callsites:
  1) Param 1:
        $\udfArg{1}$ + 1
Recursive Scenarios:
  1) Callsites: 1
     Predicate:
        SELECT NOT (2 * $\udfArg{1}$) > array_length($\udfArg{2}$, 1)
     Query:
        SELECT array_agg("RTFunc0"."v") :: int4[]
        FROM unnest($\udfArg{2}$) AS "RTFunc0"("v")
        WHERE ("RTFunc0"."v" = $\udfArg{1}$ OR "RTFunc0"."v" % $\udfArg{1}$ <> 0) 
          AND "RTFunc0"."v" = ANY(sieve($\udfArg{1}$ + 1, $\udfArg{2}$))

Basecase Scenarios:
  1) Predicate:
        SELECT (2 * $\udfArg{1}$) > array_length($\udfArg{2}$, 1)
     Query:
        SELECT array_agg("RTFunc0"."v") :: int4[]        
        FROM unnest($\udfArg{2}$) AS "RTFunc0"("v")
        WHERE True
\end{lstlisting}
    \caption{Analysis output of \autoref{udf:sieve}, for demonstration of the \RFROM-rule (slightly beautified).}
    \label{scenarios:sieve}
\end{figure}
\FloatBarrier

\subsection{\RCTE-rule}
\begin{figure}[h!]
    \centering
    \begin{lstlisting}[language={},basicstyle=\scriptsize]
### Analysis result: ###
Name:              pg_proc.fib_cte
Return type:       TAtom "numeric" [$\checkmark$ hashable]
Tail-Recursive:    $\times$
Single-Recursion:  $\times$
Parameter:
  1) TAtom "numeric" [$\checkmark$ hashable]
Constant Parameter:
   - none -
Callsites:
  1) Param 1:
        WITH v("i") AS (SELECT 1)
        SELECT (SELECT $\udfArg{1}$ - (SELECT "CTE3"."i" FROM v AS "CTE3"("i"))) 
  2) Param 1:
        SELECT (SELECT $\udfArg{1}$ - 2)
Recursive Scenarios:
  1) Callsites: 1, 2
     Predicate:
        WITH v("i") AS (SELECT 1), 
             s("i") AS (SELECT fib_cte($\udfArg{1}$ - (SELECT "CTE3"."i" FROM v AS "CTE3"("i")))), 
             t("i") AS (SELECT fib_cte($\udfArg{1}$ - 2)), 
             u("i") AS (
               SELECT "CTE4"."i" + "CTE5"."i"                    
               FROM s AS "CTE4"("i"),
                    t AS "CTE5"("i")
             ),
             p("i") AS (SELECT $\udfArg{1}$ <= 2)
        SELECT NOT (SELECT "CTE0"."i" FROM p AS "CTE0"("i"))
               AND (SELECT $\udfArg{1}$ > 0)
     Query:
        WITH v("i") AS (SELECT 1), 
             s("i") AS (SELECT fib_cte($\udfArg{1}$ - (SELECT "CTE3"."i" FROM v AS "CTE3"("i")))), 
             t("i") AS (SELECT fib_cte($\udfArg{1}$ - 2)), 
             u("i") AS (
               SELECT "CTE4"."i" + "CTE5"."i"                    
               FROM s AS "CTE4"("i"),
                    t AS "CTE5"("i")
             ),
             p("i") AS (SELECT $\udfArg{1}$ <= 2)
        SELECT (SELECT "CTE2"."i" FROM u AS "CTE2"("i")) :: numeric 

Basecase Scenarios:
  1) Predicate:
        WITH v("i") AS (SELECT 1), 
             u("i") AS (SELECT (SELECT "CTE6"."i" FROM v AS "CTE6"("i"))), 
             p("i") AS (SELECT $\udfArg{1}$ <= 2)
        SELECT (SELECT NOT $\udfArg{1}$ > 0)
     Query:
        WITH v("i") AS (SELECT 1), 
             u("i") AS (SELECT (SELECT "CTE6"."i" FROM v AS "CTE6"("i"))), 
             p("i") AS (SELECT $\udfArg{1}$ <= 2)
        SELECT (CASE WHEN (SELECT "CTE0"."i" FROM p AS "CTE0"("i"))
                       THEN (SELECT "CTE1"."i" FROM v AS "CTE1"("i"))
                     ELSE (SELECT "CTE2"."i" AS "i" FROM u AS "CTE2"("i"))
               END) :: numeric 

  2) Predicate:
        WITH v("i") AS (SELECT 1)
             p("i") AS (SELECT $\udfArg{1}$ <= 2)
        SELECT "CTE0"."i" FROM p AS "CTE0"("i")
     Query:
        WITH v("i") AS (SELECT 1)
             p("i") AS (SELECT $\udfArg{1}$ <= 2)
        SELECT (SELECT "CTE1"."i" FROM v AS "CTE1"("i")) :: numeric
\end{lstlisting}
    \caption{UDF \texttt{case\_plus\_case}, used to showcase application of the \REXPR-rule.}
    \label{udf:fib_cte}
\end{figure}

\begin{figure}[h!]
    \centering
    \begin{lstlisting}[language={},basicstyle=\scriptsize]
### Analysis result: ###
Name:              pg_proc.fib_cte
Return type:       TAtom "numeric" [$\checkmark$ hashable]
Tail-Recursive:    $\times$
Single-Recursion:  $\times$
Parameter:
  1) TAtom "numeric" [$\checkmark$ hashable]
Constant Parameter:
   - none -
Callsites:
  1) Param 1:
        WITH v("i") AS (SELECT 1)
        SELECT (SELECT $\udfArg{1}$ - (SELECT "CTE3"."i" FROM v AS "CTE3"("i"))) 
  2) Param 1:
        SELECT (SELECT $\udfArg{1}$ - 2)
Recursive Scenarios:
  1) Callsites: 1, 2
     Predicate:
        WITH v("i") AS (SELECT 1), 
             s("i") AS (SELECT fib_cte($\udfArg{1}$ - (SELECT "CTE3"."i" FROM v AS "CTE3"("i")))), 
             t("i") AS (SELECT fib_cte($\udfArg{1}$ - 2)), 
             u("i") AS (
               SELECT "CTE4"."i" + "CTE5"."i"                    
               FROM s AS "CTE4"("i"),
                    t AS "CTE5"("i")
             ),
             p("i") AS (SELECT $\udfArg{1}$ <= 2)
        SELECT NOT (SELECT "CTE0"."i" FROM p AS "CTE0"("i"))
               AND (SELECT $\udfArg{1}$ > 0)
     Query:
        WITH v("i") AS (SELECT 1), 
             s("i") AS (SELECT fib_cte($\udfArg{1}$ - (SELECT "CTE3"."i" FROM v AS "CTE3"("i")))), 
             t("i") AS (SELECT fib_cte($\udfArg{1}$ - 2)), 
             u("i") AS (
               SELECT "CTE4"."i" + "CTE5"."i"                    
               FROM s AS "CTE4"("i"),
                    t AS "CTE5"("i")
             ),
             p("i") AS (SELECT $\udfArg{1}$ <= 2)
        SELECT (SELECT "CTE2"."i" FROM u AS "CTE2"("i")) :: numeric 

Basecase Scenarios:
  1) Predicate:
        WITH v("i") AS (SELECT 1), 
             u("i") AS (SELECT (SELECT "CTE6"."i" FROM v AS "CTE6"("i"))), 
             p("i") AS (SELECT $\udfArg{1}$ <= 2)
        SELECT (SELECT NOT $\udfArg{1}$ > 0)
     Query:
        WITH v("i") AS (SELECT 1), 
             u("i") AS (SELECT (SELECT "CTE6"."i" FROM v AS "CTE6"("i"))), 
             p("i") AS (SELECT $\udfArg{1}$ <= 2)
        SELECT (CASE WHEN (SELECT "CTE0"."i" FROM p AS "CTE0"("i"))
                       THEN (SELECT "CTE1"."i" FROM v AS "CTE1"("i"))
                     ELSE (SELECT "CTE2"."i" AS "i" FROM u AS "CTE2"("i"))
               END) :: numeric 

  2) Predicate:
        WITH v("i") AS (SELECT 1)
             p("i") AS (SELECT $\udfArg{1}$ <= 2)
        SELECT "CTE0"."i" FROM p AS "CTE0"("i")
     Query:
        WITH v("i") AS (SELECT 1)
             p("i") AS (SELECT $\udfArg{1}$ <= 2)
        SELECT (SELECT "CTE1"."i" FROM v AS "CTE1"("i")) :: numeric
\end{lstlisting}
    \caption{Analysis output of \autoref{udf:case_plus_case}, for demonstration of the \RCTE- and \RWITH-rule (slightly beautified).}
    \label{scenarios:fib_cte}
\end{figure}
\FloatBarrier

\section{Example translations}

\subsection{\texttt{fib}}
\begin{figure}[h!]
    \centering\footnotesize
    \sqlcode{snippets/fib_callstack.sql}
    \caption{\texttt{callstack}-CTE as standalone query. Note that the UDF argument \texttt{\$1} is replaced with a SQL-variable \texttt{:n}}
    \label{fib:callstack_cte_complete}
\end{figure}

\section{Conditional normalization}
\begin{figure}[h!]
    \centering
\sqlcode{appendix/conditional_normalization.sql}\label{sql:conditionals}
    \caption{SQL-conditionals implemented by \texttt{CASE}-expressions.}
    \label{conditional_normalization}
\end{figure}

\begin{figure}[h!]
    \centering
    \sqlcode{appendix/paramN.sql}
    \caption{UDFs with varying number of arguments used for \autoref{fig:paramN}}
    \label{udfs:paramN}
\end{figure}

\begin{figure}[h!]
    \centering
    \footnotesize
\begin{lstlisting}[language={}]
Input:	q/fib.sql (fib(numeric))
Warning: Udf from file has replaced existing function in database!

Analysis summary:
  Callsites:		2
  Scenarios:		2
  Unhashable Args:	[]
  Return type:		Hashable

Applyed optimizations:
  [$\times$] Template for single recursive functions
  [$\times$] Template for tail recursive functions
  [$\times$] Constant parameter removal: 

### Analysis result: ###
Name:              pg_proc.fib
Return type:       TAtom "numeric" [$\checkmark$ hashable]
Tail-Recursive:    $\times$
Single-Recursion:  $\times$
Arguments:
  1) TAtom "numeric" [$\checkmark$ hashable]
Constant Arguments:
   - none -
Callsites:
  1) Param 1:
        SELECT $\udfArg{1}$ - (1) 
  2) Param 1:
        SELECT $\udfArg{1}$ - (2) 
Recursive Scenarios:
  1) Callsites: 1, 2
     Predicate:
        SELECT NOT $\udfArg{1}$ <= (2) 
     Query:
        SELECT ((fib($\udfArg{1}$ - (1)))
                +
                (fib($\udfArg{1}$ - (2)))) :: numeric AS "case" 

Basecase Scenarios:
  1) Predicate:
        SELECT $\udfArg{1}$ <= (2) 
     Query:
        SELECT (1) :: numeric AS "case" 
        
Output:	fib(numeric) to database
\end{lstlisting}
    \caption{Verbose output when translating the Fibonacci function from file \texttt{q/fib.sql} and persisting to database.}
\end{figure}
%%%%%%%%%%%%%%%%%%%%%%%%
