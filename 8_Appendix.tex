\cleardoublepage
\renewcommand{\thesection}{\Alph{section}}%
%\appendix 
%\addcontentsline{toc}{chapter}{Anhang} 

%%%%%%%%%%%%%%%%%%%%%%%%
%%%%%%%%%%%%%%%%%%%%%%%%
%%%%%%%%%%%%%%%%%%%%%%%%
\chapter[Appendix]{Appendix}
\phantomsection
\section{Examples of Rule application}
\subsection{\RWHEN-rule}
\begin{prooftree}
    \AxiomC{}\RightLabel{\scriptsize(base)}
    \UnaryInfC{\lstinputlisting{snippets/rules/fib/03-case-when.sql}}
    \AxiomC{}\RightLabel{\scriptsize(base)}
    \UnaryInfC{\lstinputlisting{snippets/rules/fib/03-case-then.sql}}
    \AxiomC{\vdots}\RightLabel{\scriptsize(else)}
    \UnaryInfC{\lstinputlisting{snippets/rules/fib/03-case-else.sql}}
    \RightLabel{\scriptsize(when)}
    \TrinaryInfC{\lstinputlisting{snippets/rules/fib/02-case.sql}}
\end{prooftree}

\section{Usage of \texttt{twr}}
\begin{figure}[h]
    \centering\footnotesize
\begin{verbatim}
Usage: twr (-f FILE | -d DIR | -u FUNC | -a) [-c] [-r] [-p] [-n FNAME/SUFFIX]
           [--user USER] [--password PASSWORD] [--host HOST] [--port PORT]
           [--database DATABASE] [-o PATH] [--disable-optimizations|--no-opt]
           [--disable-tail-recursion|--no-tr]
           [--disable-single-recursion|--no-sr]
           [--disable-constant-parameters|--no-cp] [-v] [-q|--quiet]
  Optimize recursive User-Defined Functions for PostgreSQL.

Available options:
  -f FILE                  Translate a UDF from a file
  -d DIR                   Translate UDFs that are given as single files in a
                           folder
  -u FUNC                  Translate an UDF by specifying the signature of an
                           existing function, eg. fib(int)
  -a                       Translate entire database. EXPERIMENTAL! Use with
                           caution!
  -c                       Persist translation to database
  -r                       Replace existing function in database
  -p                       Print translation(s) to stdout
  -n FNAME/SUFFIX          Name or prefix (for directory/db translation) of the
                           generated function.
  --user USER              Postgres username, default is ENV[USER]
  --password PASSWORD      Password for PostgreSQL user
  --host HOST              Host of PostgreSQL
  --port PORT              Port of PostgreSQL
  --database DATABASE      Name of a PostgreSQL database, default is USER
  -o PATH                  Create a file with the translated function.
  --disable-optimizations,--no-opt
                           Disable all optimizations
  --disable-tail-recursion,--no-tr
                           Do not use optimizations for tail recursion
  --disable-single-recursion,--no-sr
                           Do not use optimizations for single recursion
  --disable-constant-parameters,--no-cp
                           Disable optimizations for constant function
                           parameters
  -v                       Level of verbosity: -v = input/output info, -vv =
                           translation summary, -vvv = detailed analysis output
  -q,--quiet               No output
  -h,--help                Show this help text
\end{verbatim}
    \caption{Usage of the Haskell implementation}
\end{figure}

\begin{figure}[h]
    \centering
    \footnotesize
    \begin{lstlisting}
Input:	q/fib.sql (fib(numeric))
Warning: Udf from file has replaced existing function in database!

Analysis summary:
  Callsites:		2
  Scenarios:		2
  Unhashable Args:	[]
  Return type:		Hashable

Applyed optimizations:
  [✗] Template for single recursive functions
  [✗] Template for tail recursive functions
  [✗] Constant parameter removal: 

### Analysis result: ###
Name:              pg_proc.fib
Return type:       TAtom "numeric" [✔ hashable]
Tail-Recursive:    ✗
Single-Recursion:  ✗
Arguments:
  1) TAtom "numeric" [✔ hashable]
Constant Arguments:
   - none -
Callsites:
  1) Param 1:
        SELECT $1 - (1) 
  2) Param 1:
        SELECT $1 - (2) 
Recursive Scenarios:
  1) Callsites: 1, 2
     Predicate:
        SELECT NOT $1 <= (2) 
     Query:
        SELECT ((fib($1 - (1)))
                +
                (fib($1 - (2)))) :: numeric AS "case" 

Basecase Scenarios:
  1) Predicate:
        SELECT $1 <= (2) 
     Query:
        SELECT (1) :: numeric AS "case" 
        
Output:	fib(numeric) to database
    \end{lstlisting}
    \caption{Verbose output when translating the Fibonacci function from file \textit{q/fib.sql} and persisting to database.}
\end{figure}

\section{Example translations}
\subsection{\texttt{fib}}
\begin{figure}
    \centering
    \sqlcode{snippets/fib_callstack.sql}
    \caption{\texttt{callstack}-CTE as standalone query. Note that the UDF argument \texttt{\$1} is replaced with a SQL-variable \texttt{:n}}
    \label{fib:callstack_cte_complete}
\end{figure}
\subsection{\texttt{gcd} without optimization}
\subsection{\texttt{gcd} with single recursion optimization}
\subsection{\texttt{gcd} with tail recursive optimization}
\subsection{\texttt{sieve} without optimization}
\subsection{\texttt{sieve} with constant argument removal}
\section{Conditional normalization}
\section{Usage of \texttt{twr}, a Haskell implementation for PostgreSQL}
\sqlcode{appendix/conditional_normalization.sql}\label{sql:conditionals}
\section{Evaluation}
\begin{figure}
    \centering
    \sqlcode{appendix/paramN.sql}
    \caption{UDFs with varying number of arguments used for \autoref{fig:paramN}}
    \label{udfs:paramN}
\end{figure}
%%%%%%%%%%%%%%%%%%%%%%%%
